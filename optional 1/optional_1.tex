\documentclass[11pt]{article}

    \usepackage[breakable]{tcolorbox}
    \usepackage{parskip} % Stop auto-indenting (to mimic markdown behaviour)
    
    \usepackage{iftex}
    \ifPDFTeX
    	\usepackage[T1]{fontenc}
    	\usepackage{mathpazo}
    \else
    	\usepackage{fontspec}
    \fi

    % Basic figure setup, for now with no caption control since it's done
    % automatically by Pandoc (which extracts ![](path) syntax from Markdown).
    \usepackage{graphicx}
    % Maintain compatibility with old templates. Remove in nbconvert 6.0
    \let\Oldincludegraphics\includegraphics
    % Ensure that by default, figures have no caption (until we provide a
    % proper Figure object with a Caption API and a way to capture that
    % in the conversion process - todo).
    \usepackage{caption}
    \DeclareCaptionFormat{nocaption}{}
    \captionsetup{format=nocaption,aboveskip=0pt,belowskip=0pt}

    \usepackage{float}
    \floatplacement{figure}{H} % forces figures to be placed at the correct location
    \usepackage{xcolor} % Allow colors to be defined
    \usepackage{enumerate} % Needed for markdown enumerations to work
    \usepackage{geometry} % Used to adjust the document margins
    \usepackage{amsmath} % Equations
    \usepackage{amssymb} % Equations
    \usepackage{textcomp} % defines textquotesingle
    % Hack from http://tex.stackexchange.com/a/47451/13684:
    \AtBeginDocument{%
        \def\PYZsq{\textquotesingle}% Upright quotes in Pygmentized code
    }
    \usepackage{upquote} % Upright quotes for verbatim code
    \usepackage{eurosym} % defines \euro
    \usepackage[mathletters]{ucs} % Extended unicode (utf-8) support
    \usepackage{fancyvrb} % verbatim replacement that allows latex
    \usepackage{grffile} % extends the file name processing of package graphics 
                         % to support a larger range
    \makeatletter % fix for old versions of grffile with XeLaTeX
    \@ifpackagelater{grffile}{2019/11/01}
    {
      % Do nothing on new versions
    }
    {
      \def\Gread@@xetex#1{%
        \IfFileExists{"\Gin@base".bb}%
        {\Gread@eps{\Gin@base.bb}}%
        {\Gread@@xetex@aux#1}%
      }
    }
    \makeatother
    \usepackage[Export]{adjustbox} % Used to constrain images to a maximum size
    \adjustboxset{max size={0.9\linewidth}{0.9\paperheight}}

    % The hyperref package gives us a pdf with properly built
    % internal navigation ('pdf bookmarks' for the table of contents,
    % internal cross-reference links, web links for URLs, etc.)
    \usepackage{hyperref}
    % The default LaTeX title has an obnoxious amount of whitespace. By default,
    % titling removes some of it. It also provides customization options.
    \usepackage{titling}
    \usepackage{longtable} % longtable support required by pandoc >1.10
    \usepackage{booktabs}  % table support for pandoc > 1.12.2
    \usepackage[inline]{enumitem} % IRkernel/repr support (it uses the enumerate* environment)
    \usepackage[normalem]{ulem} % ulem is needed to support strikethroughs (\sout)
                                % normalem makes italics be italics, not underlines
    \usepackage{mathrsfs}
    

    
    % Colors for the hyperref package
    \definecolor{urlcolor}{rgb}{0,.145,.698}
    \definecolor{linkcolor}{rgb}{.71,0.21,0.01}
    \definecolor{citecolor}{rgb}{.12,.54,.11}

    % ANSI colors
    \definecolor{ansi-black}{HTML}{3E424D}
    \definecolor{ansi-black-intense}{HTML}{282C36}
    \definecolor{ansi-red}{HTML}{E75C58}
    \definecolor{ansi-red-intense}{HTML}{B22B31}
    \definecolor{ansi-green}{HTML}{00A250}
    \definecolor{ansi-green-intense}{HTML}{007427}
    \definecolor{ansi-yellow}{HTML}{DDB62B}
    \definecolor{ansi-yellow-intense}{HTML}{B27D12}
    \definecolor{ansi-blue}{HTML}{208FFB}
    \definecolor{ansi-blue-intense}{HTML}{0065CA}
    \definecolor{ansi-magenta}{HTML}{D160C4}
    \definecolor{ansi-magenta-intense}{HTML}{A03196}
    \definecolor{ansi-cyan}{HTML}{60C6C8}
    \definecolor{ansi-cyan-intense}{HTML}{258F8F}
    \definecolor{ansi-white}{HTML}{C5C1B4}
    \definecolor{ansi-white-intense}{HTML}{A1A6B2}
    \definecolor{ansi-default-inverse-fg}{HTML}{FFFFFF}
    \definecolor{ansi-default-inverse-bg}{HTML}{000000}

    % common color for the border for error outputs.
    \definecolor{outerrorbackground}{HTML}{FFDFDF}

    % commands and environments needed by pandoc snippets
    % extracted from the output of `pandoc -s`
    \providecommand{\tightlist}{%
      \setlength{\itemsep}{0pt}\setlength{\parskip}{0pt}}
    \DefineVerbatimEnvironment{Highlighting}{Verbatim}{commandchars=\\\{\}}
    % Add ',fontsize=\small' for more characters per line
    \newenvironment{Shaded}{}{}
    \newcommand{\KeywordTok}[1]{\textcolor[rgb]{0.00,0.44,0.13}{\textbf{{#1}}}}
    \newcommand{\DataTypeTok}[1]{\textcolor[rgb]{0.56,0.13,0.00}{{#1}}}
    \newcommand{\DecValTok}[1]{\textcolor[rgb]{0.25,0.63,0.44}{{#1}}}
    \newcommand{\BaseNTok}[1]{\textcolor[rgb]{0.25,0.63,0.44}{{#1}}}
    \newcommand{\FloatTok}[1]{\textcolor[rgb]{0.25,0.63,0.44}{{#1}}}
    \newcommand{\CharTok}[1]{\textcolor[rgb]{0.25,0.44,0.63}{{#1}}}
    \newcommand{\StringTok}[1]{\textcolor[rgb]{0.25,0.44,0.63}{{#1}}}
    \newcommand{\CommentTok}[1]{\textcolor[rgb]{0.38,0.63,0.69}{\textit{{#1}}}}
    \newcommand{\OtherTok}[1]{\textcolor[rgb]{0.00,0.44,0.13}{{#1}}}
    \newcommand{\AlertTok}[1]{\textcolor[rgb]{1.00,0.00,0.00}{\textbf{{#1}}}}
    \newcommand{\FunctionTok}[1]{\textcolor[rgb]{0.02,0.16,0.49}{{#1}}}
    \newcommand{\RegionMarkerTok}[1]{{#1}}
    \newcommand{\ErrorTok}[1]{\textcolor[rgb]{1.00,0.00,0.00}{\textbf{{#1}}}}
    \newcommand{\NormalTok}[1]{{#1}}
    
    % Additional commands for more recent versions of Pandoc
    \newcommand{\ConstantTok}[1]{\textcolor[rgb]{0.53,0.00,0.00}{{#1}}}
    \newcommand{\SpecialCharTok}[1]{\textcolor[rgb]{0.25,0.44,0.63}{{#1}}}
    \newcommand{\VerbatimStringTok}[1]{\textcolor[rgb]{0.25,0.44,0.63}{{#1}}}
    \newcommand{\SpecialStringTok}[1]{\textcolor[rgb]{0.73,0.40,0.53}{{#1}}}
    \newcommand{\ImportTok}[1]{{#1}}
    \newcommand{\DocumentationTok}[1]{\textcolor[rgb]{0.73,0.13,0.13}{\textit{{#1}}}}
    \newcommand{\AnnotationTok}[1]{\textcolor[rgb]{0.38,0.63,0.69}{\textbf{\textit{{#1}}}}}
    \newcommand{\CommentVarTok}[1]{\textcolor[rgb]{0.38,0.63,0.69}{\textbf{\textit{{#1}}}}}
    \newcommand{\VariableTok}[1]{\textcolor[rgb]{0.10,0.09,0.49}{{#1}}}
    \newcommand{\ControlFlowTok}[1]{\textcolor[rgb]{0.00,0.44,0.13}{\textbf{{#1}}}}
    \newcommand{\OperatorTok}[1]{\textcolor[rgb]{0.40,0.40,0.40}{{#1}}}
    \newcommand{\BuiltInTok}[1]{{#1}}
    \newcommand{\ExtensionTok}[1]{{#1}}
    \newcommand{\PreprocessorTok}[1]{\textcolor[rgb]{0.74,0.48,0.00}{{#1}}}
    \newcommand{\AttributeTok}[1]{\textcolor[rgb]{0.49,0.56,0.16}{{#1}}}
    \newcommand{\InformationTok}[1]{\textcolor[rgb]{0.38,0.63,0.69}{\textbf{\textit{{#1}}}}}
    \newcommand{\WarningTok}[1]{\textcolor[rgb]{0.38,0.63,0.69}{\textbf{\textit{{#1}}}}}
    
    
    % Define a nice break command that doesn't care if a line doesn't already
    % exist.
    \def\br{\hspace*{\fill} \\* }
    % Math Jax compatibility definitions
    \def\gt{>}
    \def\lt{<}
    \let\Oldtex\TeX
    \let\Oldlatex\LaTeX
    \renewcommand{\TeX}{\textrm{\Oldtex}}
    \renewcommand{\LaTeX}{\textrm{\Oldlatex}}
    % Document parameters
    % Document title
    \title{optional\_1}
    
    
    
    
    
% Pygments definitions
\makeatletter
\def\PY@reset{\let\PY@it=\relax \let\PY@bf=\relax%
    \let\PY@ul=\relax \let\PY@tc=\relax%
    \let\PY@bc=\relax \let\PY@ff=\relax}
\def\PY@tok#1{\csname PY@tok@#1\endcsname}
\def\PY@toks#1+{\ifx\relax#1\empty\else%
    \PY@tok{#1}\expandafter\PY@toks\fi}
\def\PY@do#1{\PY@bc{\PY@tc{\PY@ul{%
    \PY@it{\PY@bf{\PY@ff{#1}}}}}}}
\def\PY#1#2{\PY@reset\PY@toks#1+\relax+\PY@do{#2}}

\expandafter\def\csname PY@tok@w\endcsname{\def\PY@tc##1{\textcolor[rgb]{0.73,0.73,0.73}{##1}}}
\expandafter\def\csname PY@tok@c\endcsname{\let\PY@it=\textit\def\PY@tc##1{\textcolor[rgb]{0.25,0.50,0.50}{##1}}}
\expandafter\def\csname PY@tok@cp\endcsname{\def\PY@tc##1{\textcolor[rgb]{0.74,0.48,0.00}{##1}}}
\expandafter\def\csname PY@tok@k\endcsname{\let\PY@bf=\textbf\def\PY@tc##1{\textcolor[rgb]{0.00,0.50,0.00}{##1}}}
\expandafter\def\csname PY@tok@kp\endcsname{\def\PY@tc##1{\textcolor[rgb]{0.00,0.50,0.00}{##1}}}
\expandafter\def\csname PY@tok@kt\endcsname{\def\PY@tc##1{\textcolor[rgb]{0.69,0.00,0.25}{##1}}}
\expandafter\def\csname PY@tok@o\endcsname{\def\PY@tc##1{\textcolor[rgb]{0.40,0.40,0.40}{##1}}}
\expandafter\def\csname PY@tok@ow\endcsname{\let\PY@bf=\textbf\def\PY@tc##1{\textcolor[rgb]{0.67,0.13,1.00}{##1}}}
\expandafter\def\csname PY@tok@nb\endcsname{\def\PY@tc##1{\textcolor[rgb]{0.00,0.50,0.00}{##1}}}
\expandafter\def\csname PY@tok@nf\endcsname{\def\PY@tc##1{\textcolor[rgb]{0.00,0.00,1.00}{##1}}}
\expandafter\def\csname PY@tok@nc\endcsname{\let\PY@bf=\textbf\def\PY@tc##1{\textcolor[rgb]{0.00,0.00,1.00}{##1}}}
\expandafter\def\csname PY@tok@nn\endcsname{\let\PY@bf=\textbf\def\PY@tc##1{\textcolor[rgb]{0.00,0.00,1.00}{##1}}}
\expandafter\def\csname PY@tok@ne\endcsname{\let\PY@bf=\textbf\def\PY@tc##1{\textcolor[rgb]{0.82,0.25,0.23}{##1}}}
\expandafter\def\csname PY@tok@nv\endcsname{\def\PY@tc##1{\textcolor[rgb]{0.10,0.09,0.49}{##1}}}
\expandafter\def\csname PY@tok@no\endcsname{\def\PY@tc##1{\textcolor[rgb]{0.53,0.00,0.00}{##1}}}
\expandafter\def\csname PY@tok@nl\endcsname{\def\PY@tc##1{\textcolor[rgb]{0.63,0.63,0.00}{##1}}}
\expandafter\def\csname PY@tok@ni\endcsname{\let\PY@bf=\textbf\def\PY@tc##1{\textcolor[rgb]{0.60,0.60,0.60}{##1}}}
\expandafter\def\csname PY@tok@na\endcsname{\def\PY@tc##1{\textcolor[rgb]{0.49,0.56,0.16}{##1}}}
\expandafter\def\csname PY@tok@nt\endcsname{\let\PY@bf=\textbf\def\PY@tc##1{\textcolor[rgb]{0.00,0.50,0.00}{##1}}}
\expandafter\def\csname PY@tok@nd\endcsname{\def\PY@tc##1{\textcolor[rgb]{0.67,0.13,1.00}{##1}}}
\expandafter\def\csname PY@tok@s\endcsname{\def\PY@tc##1{\textcolor[rgb]{0.73,0.13,0.13}{##1}}}
\expandafter\def\csname PY@tok@sd\endcsname{\let\PY@it=\textit\def\PY@tc##1{\textcolor[rgb]{0.73,0.13,0.13}{##1}}}
\expandafter\def\csname PY@tok@si\endcsname{\let\PY@bf=\textbf\def\PY@tc##1{\textcolor[rgb]{0.73,0.40,0.53}{##1}}}
\expandafter\def\csname PY@tok@se\endcsname{\let\PY@bf=\textbf\def\PY@tc##1{\textcolor[rgb]{0.73,0.40,0.13}{##1}}}
\expandafter\def\csname PY@tok@sr\endcsname{\def\PY@tc##1{\textcolor[rgb]{0.73,0.40,0.53}{##1}}}
\expandafter\def\csname PY@tok@ss\endcsname{\def\PY@tc##1{\textcolor[rgb]{0.10,0.09,0.49}{##1}}}
\expandafter\def\csname PY@tok@sx\endcsname{\def\PY@tc##1{\textcolor[rgb]{0.00,0.50,0.00}{##1}}}
\expandafter\def\csname PY@tok@m\endcsname{\def\PY@tc##1{\textcolor[rgb]{0.40,0.40,0.40}{##1}}}
\expandafter\def\csname PY@tok@gh\endcsname{\let\PY@bf=\textbf\def\PY@tc##1{\textcolor[rgb]{0.00,0.00,0.50}{##1}}}
\expandafter\def\csname PY@tok@gu\endcsname{\let\PY@bf=\textbf\def\PY@tc##1{\textcolor[rgb]{0.50,0.00,0.50}{##1}}}
\expandafter\def\csname PY@tok@gd\endcsname{\def\PY@tc##1{\textcolor[rgb]{0.63,0.00,0.00}{##1}}}
\expandafter\def\csname PY@tok@gi\endcsname{\def\PY@tc##1{\textcolor[rgb]{0.00,0.63,0.00}{##1}}}
\expandafter\def\csname PY@tok@gr\endcsname{\def\PY@tc##1{\textcolor[rgb]{1.00,0.00,0.00}{##1}}}
\expandafter\def\csname PY@tok@ge\endcsname{\let\PY@it=\textit}
\expandafter\def\csname PY@tok@gs\endcsname{\let\PY@bf=\textbf}
\expandafter\def\csname PY@tok@gp\endcsname{\let\PY@bf=\textbf\def\PY@tc##1{\textcolor[rgb]{0.00,0.00,0.50}{##1}}}
\expandafter\def\csname PY@tok@go\endcsname{\def\PY@tc##1{\textcolor[rgb]{0.53,0.53,0.53}{##1}}}
\expandafter\def\csname PY@tok@gt\endcsname{\def\PY@tc##1{\textcolor[rgb]{0.00,0.27,0.87}{##1}}}
\expandafter\def\csname PY@tok@err\endcsname{\def\PY@bc##1{\setlength{\fboxsep}{0pt}\fcolorbox[rgb]{1.00,0.00,0.00}{1,1,1}{\strut ##1}}}
\expandafter\def\csname PY@tok@kc\endcsname{\let\PY@bf=\textbf\def\PY@tc##1{\textcolor[rgb]{0.00,0.50,0.00}{##1}}}
\expandafter\def\csname PY@tok@kd\endcsname{\let\PY@bf=\textbf\def\PY@tc##1{\textcolor[rgb]{0.00,0.50,0.00}{##1}}}
\expandafter\def\csname PY@tok@kn\endcsname{\let\PY@bf=\textbf\def\PY@tc##1{\textcolor[rgb]{0.00,0.50,0.00}{##1}}}
\expandafter\def\csname PY@tok@kr\endcsname{\let\PY@bf=\textbf\def\PY@tc##1{\textcolor[rgb]{0.00,0.50,0.00}{##1}}}
\expandafter\def\csname PY@tok@bp\endcsname{\def\PY@tc##1{\textcolor[rgb]{0.00,0.50,0.00}{##1}}}
\expandafter\def\csname PY@tok@fm\endcsname{\def\PY@tc##1{\textcolor[rgb]{0.00,0.00,1.00}{##1}}}
\expandafter\def\csname PY@tok@vc\endcsname{\def\PY@tc##1{\textcolor[rgb]{0.10,0.09,0.49}{##1}}}
\expandafter\def\csname PY@tok@vg\endcsname{\def\PY@tc##1{\textcolor[rgb]{0.10,0.09,0.49}{##1}}}
\expandafter\def\csname PY@tok@vi\endcsname{\def\PY@tc##1{\textcolor[rgb]{0.10,0.09,0.49}{##1}}}
\expandafter\def\csname PY@tok@vm\endcsname{\def\PY@tc##1{\textcolor[rgb]{0.10,0.09,0.49}{##1}}}
\expandafter\def\csname PY@tok@sa\endcsname{\def\PY@tc##1{\textcolor[rgb]{0.73,0.13,0.13}{##1}}}
\expandafter\def\csname PY@tok@sb\endcsname{\def\PY@tc##1{\textcolor[rgb]{0.73,0.13,0.13}{##1}}}
\expandafter\def\csname PY@tok@sc\endcsname{\def\PY@tc##1{\textcolor[rgb]{0.73,0.13,0.13}{##1}}}
\expandafter\def\csname PY@tok@dl\endcsname{\def\PY@tc##1{\textcolor[rgb]{0.73,0.13,0.13}{##1}}}
\expandafter\def\csname PY@tok@s2\endcsname{\def\PY@tc##1{\textcolor[rgb]{0.73,0.13,0.13}{##1}}}
\expandafter\def\csname PY@tok@sh\endcsname{\def\PY@tc##1{\textcolor[rgb]{0.73,0.13,0.13}{##1}}}
\expandafter\def\csname PY@tok@s1\endcsname{\def\PY@tc##1{\textcolor[rgb]{0.73,0.13,0.13}{##1}}}
\expandafter\def\csname PY@tok@mb\endcsname{\def\PY@tc##1{\textcolor[rgb]{0.40,0.40,0.40}{##1}}}
\expandafter\def\csname PY@tok@mf\endcsname{\def\PY@tc##1{\textcolor[rgb]{0.40,0.40,0.40}{##1}}}
\expandafter\def\csname PY@tok@mh\endcsname{\def\PY@tc##1{\textcolor[rgb]{0.40,0.40,0.40}{##1}}}
\expandafter\def\csname PY@tok@mi\endcsname{\def\PY@tc##1{\textcolor[rgb]{0.40,0.40,0.40}{##1}}}
\expandafter\def\csname PY@tok@il\endcsname{\def\PY@tc##1{\textcolor[rgb]{0.40,0.40,0.40}{##1}}}
\expandafter\def\csname PY@tok@mo\endcsname{\def\PY@tc##1{\textcolor[rgb]{0.40,0.40,0.40}{##1}}}
\expandafter\def\csname PY@tok@ch\endcsname{\let\PY@it=\textit\def\PY@tc##1{\textcolor[rgb]{0.25,0.50,0.50}{##1}}}
\expandafter\def\csname PY@tok@cm\endcsname{\let\PY@it=\textit\def\PY@tc##1{\textcolor[rgb]{0.25,0.50,0.50}{##1}}}
\expandafter\def\csname PY@tok@cpf\endcsname{\let\PY@it=\textit\def\PY@tc##1{\textcolor[rgb]{0.25,0.50,0.50}{##1}}}
\expandafter\def\csname PY@tok@c1\endcsname{\let\PY@it=\textit\def\PY@tc##1{\textcolor[rgb]{0.25,0.50,0.50}{##1}}}
\expandafter\def\csname PY@tok@cs\endcsname{\let\PY@it=\textit\def\PY@tc##1{\textcolor[rgb]{0.25,0.50,0.50}{##1}}}

\def\PYZbs{\char`\\}
\def\PYZus{\char`\_}
\def\PYZob{\char`\{}
\def\PYZcb{\char`\}}
\def\PYZca{\char`\^}
\def\PYZam{\char`\&}
\def\PYZlt{\char`\<}
\def\PYZgt{\char`\>}
\def\PYZsh{\char`\#}
\def\PYZpc{\char`\%}
\def\PYZdl{\char`\$}
\def\PYZhy{\char`\-}
\def\PYZsq{\char`\'}
\def\PYZdq{\char`\"}
\def\PYZti{\char`\~}
% for compatibility with earlier versions
\def\PYZat{@}
\def\PYZlb{[}
\def\PYZrb{]}
\makeatother


    % For linebreaks inside Verbatim environment from package fancyvrb. 
    \makeatletter
        \newbox\Wrappedcontinuationbox 
        \newbox\Wrappedvisiblespacebox 
        \newcommand*\Wrappedvisiblespace {\textcolor{red}{\textvisiblespace}} 
        \newcommand*\Wrappedcontinuationsymbol {\textcolor{red}{\llap{\tiny$\m@th\hookrightarrow$}}} 
        \newcommand*\Wrappedcontinuationindent {3ex } 
        \newcommand*\Wrappedafterbreak {\kern\Wrappedcontinuationindent\copy\Wrappedcontinuationbox} 
        % Take advantage of the already applied Pygments mark-up to insert 
        % potential linebreaks for TeX processing. 
        %        {, <, #, %, $, ' and ": go to next line. 
        %        _, }, ^, &, >, - and ~: stay at end of broken line. 
        % Use of \textquotesingle for straight quote. 
        \newcommand*\Wrappedbreaksatspecials {% 
            \def\PYGZus{\discretionary{\char`\_}{\Wrappedafterbreak}{\char`\_}}% 
            \def\PYGZob{\discretionary{}{\Wrappedafterbreak\char`\{}{\char`\{}}% 
            \def\PYGZcb{\discretionary{\char`\}}{\Wrappedafterbreak}{\char`\}}}% 
            \def\PYGZca{\discretionary{\char`\^}{\Wrappedafterbreak}{\char`\^}}% 
            \def\PYGZam{\discretionary{\char`\&}{\Wrappedafterbreak}{\char`\&}}% 
            \def\PYGZlt{\discretionary{}{\Wrappedafterbreak\char`\<}{\char`\<}}% 
            \def\PYGZgt{\discretionary{\char`\>}{\Wrappedafterbreak}{\char`\>}}% 
            \def\PYGZsh{\discretionary{}{\Wrappedafterbreak\char`\#}{\char`\#}}% 
            \def\PYGZpc{\discretionary{}{\Wrappedafterbreak\char`\%}{\char`\%}}% 
            \def\PYGZdl{\discretionary{}{\Wrappedafterbreak\char`\$}{\char`\$}}% 
            \def\PYGZhy{\discretionary{\char`\-}{\Wrappedafterbreak}{\char`\-}}% 
            \def\PYGZsq{\discretionary{}{\Wrappedafterbreak\textquotesingle}{\textquotesingle}}% 
            \def\PYGZdq{\discretionary{}{\Wrappedafterbreak\char`\"}{\char`\"}}% 
            \def\PYGZti{\discretionary{\char`\~}{\Wrappedafterbreak}{\char`\~}}% 
        } 
        % Some characters . , ; ? ! / are not pygmentized. 
        % This macro makes them "active" and they will insert potential linebreaks 
        \newcommand*\Wrappedbreaksatpunct {% 
            \lccode`\~`\.\lowercase{\def~}{\discretionary{\hbox{\char`\.}}{\Wrappedafterbreak}{\hbox{\char`\.}}}% 
            \lccode`\~`\,\lowercase{\def~}{\discretionary{\hbox{\char`\,}}{\Wrappedafterbreak}{\hbox{\char`\,}}}% 
            \lccode`\~`\;\lowercase{\def~}{\discretionary{\hbox{\char`\;}}{\Wrappedafterbreak}{\hbox{\char`\;}}}% 
            \lccode`\~`\:\lowercase{\def~}{\discretionary{\hbox{\char`\:}}{\Wrappedafterbreak}{\hbox{\char`\:}}}% 
            \lccode`\~`\?\lowercase{\def~}{\discretionary{\hbox{\char`\?}}{\Wrappedafterbreak}{\hbox{\char`\?}}}% 
            \lccode`\~`\!\lowercase{\def~}{\discretionary{\hbox{\char`\!}}{\Wrappedafterbreak}{\hbox{\char`\!}}}% 
            \lccode`\~`\/\lowercase{\def~}{\discretionary{\hbox{\char`\/}}{\Wrappedafterbreak}{\hbox{\char`\/}}}% 
            \catcode`\.\active
            \catcode`\,\active 
            \catcode`\;\active
            \catcode`\:\active
            \catcode`\?\active
            \catcode`\!\active
            \catcode`\/\active 
            \lccode`\~`\~ 	
        }
    \makeatother

    \let\OriginalVerbatim=\Verbatim
    \makeatletter
    \renewcommand{\Verbatim}[1][1]{%
        %\parskip\z@skip
        \sbox\Wrappedcontinuationbox {\Wrappedcontinuationsymbol}%
        \sbox\Wrappedvisiblespacebox {\FV@SetupFont\Wrappedvisiblespace}%
        \def\FancyVerbFormatLine ##1{\hsize\linewidth
            \vtop{\raggedright\hyphenpenalty\z@\exhyphenpenalty\z@
                \doublehyphendemerits\z@\finalhyphendemerits\z@
                \strut ##1\strut}%
        }%
        % If the linebreak is at a space, the latter will be displayed as visible
        % space at end of first line, and a continuation symbol starts next line.
        % Stretch/shrink are however usually zero for typewriter font.
        \def\FV@Space {%
            \nobreak\hskip\z@ plus\fontdimen3\font minus\fontdimen4\font
            \discretionary{\copy\Wrappedvisiblespacebox}{\Wrappedafterbreak}
            {\kern\fontdimen2\font}%
        }%
        
        % Allow breaks at special characters using \PYG... macros.
        \Wrappedbreaksatspecials
        % Breaks at punctuation characters . , ; ? ! and / need catcode=\active 	
        \OriginalVerbatim[#1,codes*=\Wrappedbreaksatpunct]%
    }
    \makeatother

    % Exact colors from NB
    \definecolor{incolor}{HTML}{303F9F}
    \definecolor{outcolor}{HTML}{D84315}
    \definecolor{cellborder}{HTML}{CFCFCF}
    \definecolor{cellbackground}{HTML}{F7F7F7}
    
    % prompt
    \makeatletter
    \newcommand{\boxspacing}{\kern\kvtcb@left@rule\kern\kvtcb@boxsep}
    \makeatother
    \newcommand{\prompt}[4]{
        {\ttfamily\llap{{\color{#2}[#3]:\hspace{3pt}#4}}\vspace{-\baselineskip}}
    }
    

    
    % Prevent overflowing lines due to hard-to-break entities
    \sloppy 
    % Setup hyperref package
    \hypersetup{
      breaklinks=true,  % so long urls are correctly broken across lines
      colorlinks=true,
      urlcolor=urlcolor,
      linkcolor=linkcolor,
      citecolor=citecolor,
      }
    % Slightly bigger margins than the latex defaults
    
    \geometry{verbose,tmargin=1in,bmargin=1in,lmargin=1in,rmargin=1in}
    
    

\begin{document}
    
    \maketitle
    
    

    
    \hypertarget{write-a-function-that-inputs-a-number-and-prints-the-multiplication-table-of-that-number}{%
\section{1. Write a function that inputs a number and prints the
multiplication table of that
number}\label{write-a-function-that-inputs-a-number-and-prints-the-multiplication-table-of-that-number}}

    \begin{tcolorbox}[breakable, size=fbox, boxrule=1pt, pad at break*=1mm,colback=cellbackground, colframe=cellborder]
\prompt{In}{incolor}{1}{\boxspacing}
\begin{Verbatim}[commandchars=\\\{\}]
\PY{c+c1}{\PYZsh{}function for printing multiplication table of input number}
\PY{k}{def} \PY{n+nf}{multiplicationTable}\PY{p}{(}\PY{p}{)}\PY{p}{:}
    \PY{c+c1}{\PYZsh{} for taking input from user in integer format}
    \PY{n}{number} \PY{o}{=} \PY{n+nb}{int}\PY{p}{(}\PY{n+nb}{input}\PY{p}{(}\PY{p}{)}\PY{p}{)}
    \PY{c+c1}{\PYZsh{}for printing multiplication table for input number}
    \PY{n+nb}{print}\PY{p}{(}\PY{l+s+s2}{\PYZdq{}}\PY{l+s+s2}{\PYZhy{}\PYZhy{}Multiplication table of}\PY{l+s+s2}{\PYZdq{}}\PY{p}{,}\PY{n}{number}\PY{p}{,} \PY{l+s+s2}{\PYZdq{}}\PY{l+s+s2}{is\PYZhy{}\PYZhy{}}\PY{l+s+s2}{\PYZdq{}}\PY{p}{)}
    \PY{k}{for} \PY{n}{i} \PY{o+ow}{in} \PY{n+nb}{range}\PY{p}{(}\PY{l+m+mi}{1}\PY{p}{,} \PY{l+m+mi}{11}\PY{p}{)}\PY{p}{:}
       \PY{n+nb}{print}\PY{p}{(}\PY{n}{number}\PY{p}{,} \PY{l+s+s1}{\PYZsq{}}\PY{l+s+s1}{x}\PY{l+s+s1}{\PYZsq{}}\PY{p}{,} \PY{n}{i}\PY{p}{,} \PY{l+s+s1}{\PYZsq{}}\PY{l+s+s1}{=}\PY{l+s+s1}{\PYZsq{}}\PY{p}{,} \PY{n}{number} \PY{o}{*} \PY{n}{i} \PY{p}{)}
\PY{n}{multiplicationTable}\PY{p}{(}\PY{p}{)}
\end{Verbatim}
\end{tcolorbox}

    \begin{Verbatim}[commandchars=\\\{\}]
2
--Multiplication table of 2 is--
2 x 1 = 2
2 x 2 = 4
2 x 3 = 6
2 x 4 = 8
2 x 5 = 10
2 x 6 = 12
2 x 7 = 14
2 x 8 = 16
2 x 9 = 18
2 x 10 = 20
    \end{Verbatim}

    \hypertarget{write-a-program-to-print-twin-primes-less-than-1000.-if-two-consecutive-odd-numbers-are-both-prime-then-they-are-known-as-twin-primes}{%
\section{2. Write a program to print twin primes less than 1000. If two
consecutive odd numbers are both prime then they are known as twin
primes}\label{write-a-program-to-print-twin-primes-less-than-1000.-if-two-consecutive-odd-numbers-are-both-prime-then-they-are-known-as-twin-primes}}

    \begin{tcolorbox}[breakable, size=fbox, boxrule=1pt, pad at break*=1mm,colback=cellbackground, colframe=cellborder]
\prompt{In}{incolor}{10}{\boxspacing}
\begin{Verbatim}[commandchars=\\\{\}]
\PY{k}{def} \PY{n+nf}{isPrime}\PY{p}{(}\PY{n}{n}\PY{p}{)}\PY{p}{:}
    \PY{k}{for} \PY{n}{i} \PY{o+ow}{in} \PY{n+nb}{range}\PY{p}{(}\PY{l+m+mi}{2}\PY{p}{,}\PY{n}{n}\PY{p}{)}\PY{p}{:}
        \PY{k}{if}\PY{p}{(}\PY{n}{n}\PY{o}{\PYZpc{}}\PY{k}{i}==0):
            \PY{k}{return} \PY{k+kc}{False}
    \PY{k}{return} \PY{k+kc}{True}
\PY{n+nb}{print}\PY{p}{(}\PY{l+s+s2}{\PYZdq{}}\PY{l+s+s2}{Twin Primes between 1 and 1000 are : \PYZhy{}}\PY{l+s+s2}{\PYZdq{}}\PY{p}{)}    
\PY{k}{for} \PY{n}{i} \PY{o+ow}{in} \PY{n+nb}{range}\PY{p}{(}\PY{l+m+mi}{2}\PY{p}{,}\PY{l+m+mi}{1000}\PY{p}{)}\PY{p}{:}
    \PY{k}{if}\PY{p}{(}\PY{n}{isPrime}\PY{p}{(}\PY{n}{i}\PY{p}{)} \PY{o+ow}{and} \PY{n}{isPrime}\PY{p}{(}\PY{n}{i}\PY{o}{+}\PY{l+m+mi}{2}\PY{p}{)}\PY{p}{)}\PY{p}{:}
        \PY{n+nb}{print} \PY{p}{(}\PY{n}{i}\PY{p}{,}\PY{l+s+s1}{\PYZsq{}}\PY{l+s+s1}{,}\PY{l+s+s1}{\PYZsq{}}\PY{p}{,}\PY{n}{i}\PY{o}{+}\PY{l+m+mi}{2}\PY{p}{)}
\end{Verbatim}
\end{tcolorbox}

    \begin{Verbatim}[commandchars=\\\{\}]
Twin Primes between 1 and 1000 are : -
3 , 5
5 , 7
11 , 13
17 , 19
29 , 31
41 , 43
59 , 61
71 , 73
101 , 103
107 , 109
137 , 139
149 , 151
179 , 181
191 , 193
197 , 199
227 , 229
239 , 241
269 , 271
281 , 283
311 , 313
347 , 349
419 , 421
431 , 433
461 , 463
521 , 523
569 , 571
599 , 601
617 , 619
641 , 643
659 , 661
809 , 811
821 , 823
827 , 829
857 , 859
881 , 883
    \end{Verbatim}

    \hypertarget{write-a-program-to-find-out-the-prime-factors-of-a-number.-example-prime-factors-of-56---2-2-2-7}{%
\section{3. Write a program to find out the prime factors of a number.
Example: prime factors of 56 :- 2, 2, 2,
7}\label{write-a-program-to-find-out-the-prime-factors-of-a-number.-example-prime-factors-of-56---2-2-2-7}}

    \begin{tcolorbox}[breakable, size=fbox, boxrule=1pt, pad at break*=1mm,colback=cellbackground, colframe=cellborder]
\prompt{In}{incolor}{11}{\boxspacing}
\begin{Verbatim}[commandchars=\\\{\}]
\PY{c+c1}{\PYZsh{}function for printing prime factors of a number }
\PY{k}{def} \PY{n+nf}{prime\PYZus{}factors}\PY{p}{(}\PY{n}{n}\PY{p}{)}\PY{p}{:} 
    \PY{k}{while} \PY{n}{n} \PY{o}{\PYZpc{}} \PY{l+m+mi}{2} \PY{o}{==} \PY{l+m+mi}{0}\PY{p}{:} 
        \PY{n+nb}{print} \PY{p}{(}\PY{l+m+mi}{2}\PY{p}{)}\PY{p}{,} 
        \PY{n}{n} \PY{o}{=} \PY{n}{n} \PY{o}{/} \PY{l+m+mi}{2}
    \PY{k}{for} \PY{n}{i} \PY{o+ow}{in} \PY{n+nb}{range}\PY{p}{(}\PY{l+m+mi}{3}\PY{p}{,}\PY{n+nb}{int}\PY{p}{(}\PY{n}{n} \PY{o}{*}\PY{o}{*} \PY{l+m+mf}{0.5}\PY{p}{)}\PY{o}{+}\PY{l+m+mi}{1}\PY{p}{,}\PY{l+m+mi}{2}\PY{p}{)}\PY{p}{:} 
        \PY{k}{while} \PY{n}{n} \PY{o}{\PYZpc{}} \PY{n}{i}\PY{o}{==} \PY{l+m+mi}{0}\PY{p}{:} 
            \PY{n+nb}{print} \PY{p}{(}\PY{n}{i}\PY{p}{)}\PY{p}{,} 
            \PY{n}{n} \PY{o}{=} \PY{n}{n} \PY{o}{/} \PY{n}{i} 
    \PY{k}{if} \PY{n}{n} \PY{o}{\PYZgt{}} \PY{l+m+mi}{2}\PY{p}{:} 
        \PY{n+nb}{print} \PY{p}{(}\PY{n}{n}\PY{p}{)} 
   
\PY{n}{m} \PY{o}{=} \PY{n+nb}{int}\PY{p}{(}\PY{n+nb}{input}\PY{p}{(}\PY{p}{)}\PY{p}{)} \PY{c+c1}{\PYZsh{}For taking input from user}
\PY{n+nb}{print}\PY{p}{(}\PY{l+s+s2}{\PYZdq{}}\PY{l+s+s2}{Prime Factors of}\PY{l+s+s2}{\PYZdq{}}\PY{p}{,}\PY{n}{m}\PY{p}{,}\PY{l+s+s2}{\PYZdq{}}\PY{l+s+s2}{are \PYZhy{} }\PY{l+s+s2}{\PYZdq{}}\PY{p}{)}
\PY{n}{prime\PYZus{}factors}\PY{p}{(}\PY{n}{m}\PY{p}{)}
\end{Verbatim}
\end{tcolorbox}

    \begin{Verbatim}[commandchars=\\\{\}]
56
Prime Factors of 56 are -
2
2
2
7.0
    \end{Verbatim}

    \hypertarget{write-a-program-to-implement-these-formulae-of-permutations-and-combinations.-number-of-permutations-of-n-objects-taken-r-at-a-time-pn-r-n-n-r.-number-of-combinations-of-n-objects-taken-r-at-a-time-is-cn-r-n-rn-r-pnr-r}{%
\section{4. Write a program to implement these formulae of permutations
and combinations. Number of permutations of n objects taken r at a time:
p(n, r) = n! / (n-r)!. Number of combinations of n objects taken r at a
time is: c(n, r) = n! / (r!*(n-r)!) = p(n,r) /
r!}\label{write-a-program-to-implement-these-formulae-of-permutations-and-combinations.-number-of-permutations-of-n-objects-taken-r-at-a-time-pn-r-n-n-r.-number-of-combinations-of-n-objects-taken-r-at-a-time-is-cn-r-n-rn-r-pnr-r}}

    \begin{tcolorbox}[breakable, size=fbox, boxrule=1pt, pad at break*=1mm,colback=cellbackground, colframe=cellborder]
\prompt{In}{incolor}{12}{\boxspacing}
\begin{Verbatim}[commandchars=\\\{\}]
\PY{c+c1}{\PYZsh{}function for returning factorial of a number}
\PY{k}{def} \PY{n+nf}{factorial}\PY{p}{(}\PY{n}{n}\PY{p}{)}\PY{p}{:} 
    \PY{n}{fact} \PY{o}{=} \PY{l+m+mi}{1}\PY{p}{;} 
    \PY{k}{for} \PY{n}{i} \PY{o+ow}{in} \PY{n+nb}{range}\PY{p}{(}\PY{l+m+mi}{2}\PY{p}{,}\PY{n}{n}\PY{o}{+}\PY{l+m+mi}{1}\PY{p}{)}\PY{p}{:} 
        \PY{n}{fact} \PY{o}{=} \PY{n}{fact} \PY{o}{*} \PY{n}{i}\PY{p}{;} 
    \PY{k}{return} \PY{n}{fact}\PY{p}{;} 
\PY{c+c1}{\PYZsh{}function for returning Number of permutations of n objects taken r at a time}
\PY{k}{def} \PY{n+nf}{nPr}\PY{p}{(}\PY{n}{n}\PY{p}{,} \PY{n}{r}\PY{p}{)}\PY{p}{:} 
    \PY{n}{pnr} \PY{o}{=} \PY{n}{factorial}\PY{p}{(}\PY{n}{n}\PY{p}{)} \PY{o}{/} \PY{n}{factorial}\PY{p}{(}\PY{n}{n} \PY{o}{\PYZhy{}} \PY{n}{r}\PY{p}{)}\PY{p}{;} 
    \PY{k}{return} \PY{n}{pnr}\PY{p}{;}  
\PY{c+c1}{\PYZsh{}function for returning Number of combinations of n objects taken r at a time}
\PY{k}{def} \PY{n+nf}{nCr}\PY{p}{(}\PY{n}{n}\PY{p}{,} \PY{n}{r}\PY{p}{)}\PY{p}{:} 
    \PY{k}{return} \PY{p}{(}\PY{n}{factorial}\PY{p}{(}\PY{n}{n}\PY{p}{)} \PY{o}{/} \PY{p}{(}\PY{n}{factorial}\PY{p}{(}\PY{n}{r}\PY{p}{)} \PY{o}{*} \PY{n}{factorial}\PY{p}{(}\PY{n}{n} \PY{o}{\PYZhy{}} \PY{n}{r}\PY{p}{)}\PY{p}{)}\PY{p}{)} 

\PY{n+nb}{print}\PY{p}{(}\PY{l+s+s2}{\PYZdq{}}\PY{l+s+s2}{Number of permutations of 8 objects, taken 5 at a time is }\PY{l+s+s2}{\PYZdq{}}\PY{p}{,}\PY{n}{nPr}\PY{p}{(}\PY{l+m+mi}{8}\PY{p}{,} \PY{l+m+mi}{5}\PY{p}{)}\PY{p}{)}
\PY{n+nb}{print}\PY{p}{(}\PY{l+s+s2}{\PYZdq{}}\PY{l+s+s2}{Number of combinations of 8 objects, taken 5 at a time is }\PY{l+s+s2}{\PYZdq{}}\PY{p}{,}\PY{n}{nCr}\PY{p}{(}\PY{l+m+mi}{8}\PY{p}{,} \PY{l+m+mi}{5}\PY{p}{)}\PY{p}{)}
\end{Verbatim}
\end{tcolorbox}

    \begin{Verbatim}[commandchars=\\\{\}]
Number of permutations of 8 objects, taken 5 at a time is  6720.0
Number of combinations of 8 objects, taken 5 at a time is  56.0
    \end{Verbatim}

    \hypertarget{write-a-function-that-converts-a-decimal-number-to-binary-number}{%
\section{5. Write a function that converts a decimal number to binary
number}\label{write-a-function-that-converts-a-decimal-number-to-binary-number}}

    \begin{tcolorbox}[breakable, size=fbox, boxrule=1pt, pad at break*=1mm,colback=cellbackground, colframe=cellborder]
\prompt{In}{incolor}{13}{\boxspacing}
\begin{Verbatim}[commandchars=\\\{\}]
\PY{k}{def} \PY{n+nf}{decimalToBinary}\PY{p}{(}\PY{n}{n}\PY{p}{)}\PY{p}{:} 
    \PY{n}{binaryNum} \PY{o}{=} \PY{p}{[}\PY{l+m+mi}{0}\PY{p}{]} \PY{o}{*} \PY{n}{n} 
    \PY{n}{i} \PY{o}{=} \PY{l+m+mi}{0}
    \PY{k}{while} \PY{p}{(}\PY{n}{n} \PY{o}{\PYZgt{}} \PY{l+m+mi}{0}\PY{p}{)}\PY{p}{:}  
        \PY{n}{binaryNum}\PY{p}{[}\PY{n}{i}\PY{p}{]} \PY{o}{=} \PY{n}{n} \PY{o}{\PYZpc{}} \PY{l+m+mi}{2} 
        \PY{n}{n} \PY{o}{=} \PY{n+nb}{int}\PY{p}{(}\PY{n}{n} \PY{o}{/} \PY{l+m+mi}{2}\PY{p}{)}
        \PY{n}{i} \PY{o}{+}\PY{o}{=} \PY{l+m+mi}{1}
    \PY{k}{for} \PY{n}{j} \PY{o+ow}{in} \PY{n+nb}{range}\PY{p}{(}\PY{n}{i} \PY{o}{\PYZhy{}} \PY{l+m+mi}{1}\PY{p}{,} \PY{o}{\PYZhy{}}\PY{l+m+mi}{1}\PY{p}{,} \PY{o}{\PYZhy{}}\PY{l+m+mi}{1}\PY{p}{)}\PY{p}{:} 
        \PY{n+nb}{print}\PY{p}{(}\PY{n}{binaryNum}\PY{p}{[}\PY{n}{j}\PY{p}{]}\PY{p}{,} \PY{n}{end} \PY{o}{=} \PY{l+s+s2}{\PYZdq{}}\PY{l+s+s2}{\PYZdq{}}\PY{p}{)}
\PY{n+nb}{print}\PY{p}{(}\PY{l+s+s2}{\PYZdq{}}\PY{l+s+s2}{Binary number of 45 is \PYZhy{} }\PY{l+s+s2}{\PYZdq{}}\PY{p}{,}\PY{n}{end}\PY{o}{=}\PY{l+s+s2}{\PYZdq{}}\PY{l+s+s2}{\PYZdq{}}\PY{p}{)}
\PY{n}{decimalToBinary}\PY{p}{(}\PY{l+m+mi}{45}\PY{p}{)}
\end{Verbatim}
\end{tcolorbox}

    \begin{Verbatim}[commandchars=\\\{\}]
Binary number of 45 is - 101101
    \end{Verbatim}

    \hypertarget{write-a-function-cubesum-that-accepts-an-integer-and-returns-the-sum-of-the-cubes-of-individual-digits-of-that-number.-use-this-function-to-make-functions-printarmstrong-and-isarmstrong-to-print-armstrong-numbers-and-to-find-whether-is-an-armstrong-number}{%
\section{6. Write a function cubesum() that accepts an integer and
returns the sum of the cubes of individual digits of that number. Use
this function to make functions PrintArmstrong() and isArmstrong() to
print Armstrong numbers and to find whether is an Armstrong
number}\label{write-a-function-cubesum-that-accepts-an-integer-and-returns-the-sum-of-the-cubes-of-individual-digits-of-that-number.-use-this-function-to-make-functions-printarmstrong-and-isarmstrong-to-print-armstrong-numbers-and-to-find-whether-is-an-armstrong-number}}

    \begin{tcolorbox}[breakable, size=fbox, boxrule=1pt, pad at break*=1mm,colback=cellbackground, colframe=cellborder]
\prompt{In}{incolor}{14}{\boxspacing}
\begin{Verbatim}[commandchars=\\\{\}]
\PY{c+c1}{\PYZsh{}function to add cube of digits}
\PY{k}{def} \PY{n+nf}{cubesum}\PY{p}{(}\PY{n}{n}\PY{p}{)}\PY{p}{:}
    \PY{n}{temp} \PY{o}{=} \PY{n}{n}
    \PY{n+nb}{sum} \PY{o}{=} \PY{l+m+mi}{0}
    \PY{k}{while} \PY{n}{temp} \PY{o}{\PYZgt{}} \PY{l+m+mi}{0}\PY{p}{:}
        \PY{n}{digit} \PY{o}{=} \PY{n}{temp} \PY{o}{\PYZpc{}} \PY{l+m+mi}{10}
        \PY{n+nb}{sum} \PY{o}{+}\PY{o}{=} \PY{n}{digit} \PY{o}{*}\PY{o}{*} \PY{l+m+mi}{3}
        \PY{n}{temp} \PY{o}{/}\PY{o}{/}\PY{o}{=} \PY{l+m+mi}{10}
    \PY{k}{return} \PY{n+nb}{sum}

\PY{k}{def} \PY{n+nf}{isArmstrong}\PY{p}{(}\PY{n}{x}\PY{p}{)}\PY{p}{:}
    \PY{k}{if} \PY{p}{(}\PY{n}{x} \PY{o}{==} \PY{n}{cubesum}\PY{p}{(}\PY{n}{x}\PY{p}{)}\PY{p}{)}\PY{p}{:}
        \PY{k}{return} \PY{k+kc}{True}
    \PY{k}{else}\PY{p}{:}
        \PY{k}{return} \PY{k+kc}{False}

\PY{k}{def} \PY{n+nf}{PrintArmstrong}\PY{p}{(}\PY{n+nb}{min}\PY{p}{,}\PY{n+nb}{max}\PY{p}{)}\PY{p}{:}
    \PY{k}{for} \PY{n}{i} \PY{o+ow}{in} \PY{n+nb}{range}\PY{p}{(}\PY{n+nb}{min}\PY{p}{,}\PY{n+nb}{max}\PY{p}{)}\PY{p}{:}
        \PY{k}{if}\PY{p}{(}\PY{n}{isArmstrong}\PY{p}{(}\PY{n}{i}\PY{p}{)}\PY{p}{)}\PY{p}{:}
            \PY{n+nb}{print}\PY{p}{(}\PY{n}{i}\PY{p}{)}
\PY{n+nb}{print}\PY{p}{(}\PY{l+s+s2}{\PYZdq{}}\PY{l+s+s2}{Armstrong number in range 1\PYZhy{}1000 are :\PYZhy{}}\PY{l+s+s2}{\PYZdq{}}\PY{p}{)}
\PY{n}{PrintArmstrong}\PY{p}{(}\PY{l+m+mi}{1}\PY{p}{,}\PY{l+m+mi}{1000}\PY{p}{)}
\end{Verbatim}
\end{tcolorbox}

    \begin{Verbatim}[commandchars=\\\{\}]
Armstrong number in range 1-1000 are :-
1
153
370
371
407
    \end{Verbatim}

    \hypertarget{write-a-function-proddigits-that-inputs-a-number-and-returns-the-product-of-digits-of-that-number.}{%
\section{7. Write a function prodDigits() that inputs a number and
returns the product of digits of that
number.}\label{write-a-function-proddigits-that-inputs-a-number-and-returns-the-product-of-digits-of-that-number.}}

    \begin{tcolorbox}[breakable, size=fbox, boxrule=1pt, pad at break*=1mm,colback=cellbackground, colframe=cellborder]
\prompt{In}{incolor}{15}{\boxspacing}
\begin{Verbatim}[commandchars=\\\{\}]
\PY{c+c1}{\PYZsh{} Function to get product of digits }
\PY{k}{def} \PY{n+nf}{prodDigits}\PY{p}{(}\PY{n}{n}\PY{p}{)}\PY{p}{:}
    \PY{n}{product} \PY{o}{=} \PY{l+m+mi}{1}
    \PY{k}{while} \PY{p}{(}\PY{n}{n} \PY{o}{!=} \PY{l+m+mi}{0}\PY{p}{)}\PY{p}{:} 
        \PY{n}{product} \PY{o}{=} \PY{n}{product} \PY{o}{*} \PY{p}{(}\PY{n}{n} \PY{o}{\PYZpc{}} \PY{l+m+mi}{10}\PY{p}{)} 
        \PY{n}{n} \PY{o}{=} \PY{n}{n} \PY{o}{/}\PY{o}{/} \PY{l+m+mi}{10}
    \PY{k}{return} \PY{n}{product} 
\PY{n+nb}{print}\PY{p}{(}\PY{l+s+s2}{\PYZdq{}}\PY{l+s+s2}{Product of digits of number 4671 is :\PYZhy{}}\PY{l+s+s2}{\PYZdq{}}\PY{p}{,}\PY{n}{prodDigits}\PY{p}{(}\PY{l+m+mi}{4671}\PY{p}{)}\PY{p}{)} 
\end{Verbatim}
\end{tcolorbox}

    \begin{Verbatim}[commandchars=\\\{\}]
Product of digits of number 4671 is :- 168
    \end{Verbatim}

    \hypertarget{if-all-digits-of-a-number-n-are-multiplied-by-each-other-repeating-with-the-product-the-one-digit-number-obtained-at-last-is-called-the-multiplicative-digital-root-of-n.-the-number-of-times-digits-need-to-be-multiplied-to-reach-one-digit-is-called-the-multiplicative-persistance-of-n.-example-86---48---32---6-mdr-6-mpersistence-3-341---12-2-mdr-2-mpersistence-2-using-the-function-proddigits-of-previous-exercise-write-functions-mdr-and-mpersistence-that-input-a-number-and-return-its-multiplicative-digital-root-and-multiplicative-persistence-respectively}{%
\section{8. If all digits of a number n are multiplied by each other
repeating with the product, the one digit number obtained at last is
called the multiplicative digital root of n.~The number of times digits
need to be multiplied to reach one digit is called the multiplicative
persistance of n.~Example: 86 -\textgreater{} 48 -\textgreater{} 32
-\textgreater{} 6 (MDR 6, MPersistence 3) 341 -\textgreater{}
12-\textgreater2 (MDR 2, MPersistence 2) Using the function prodDigits()
of previous exercise write functions MDR() and MPersistence() that input
a number and return its multiplicative digital root and multiplicative
persistence
respectively}\label{if-all-digits-of-a-number-n-are-multiplied-by-each-other-repeating-with-the-product-the-one-digit-number-obtained-at-last-is-called-the-multiplicative-digital-root-of-n.-the-number-of-times-digits-need-to-be-multiplied-to-reach-one-digit-is-called-the-multiplicative-persistance-of-n.-example-86---48---32---6-mdr-6-mpersistence-3-341---12-2-mdr-2-mpersistence-2-using-the-function-proddigits-of-previous-exercise-write-functions-mdr-and-mpersistence-that-input-a-number-and-return-its-multiplicative-digital-root-and-multiplicative-persistence-respectively}}

    \begin{tcolorbox}[breakable, size=fbox, boxrule=1pt, pad at break*=1mm,colback=cellbackground, colframe=cellborder]
\prompt{In}{incolor}{16}{\boxspacing}
\begin{Verbatim}[commandchars=\\\{\}]
\PY{k}{def} \PY{n+nf}{MDR}\PY{p}{(}\PY{n}{n}\PY{p}{)}\PY{p}{:}
    \PY{n}{temp} \PY{o}{=} \PY{n+nb}{str}\PY{p}{(}\PY{n}{n}\PY{p}{)}
    \PY{k}{while} \PY{n+nb}{len}\PY{p}{(}\PY{n}{temp}\PY{p}{)} \PY{o}{\PYZgt{}} \PY{l+m+mi}{1}\PY{p}{:}
        \PY{n}{temp} \PY{o}{=} \PY{n+nb}{str}\PY{p}{(}\PY{n}{prodDigits}\PY{p}{(}\PY{n+nb}{int}\PY{p}{(}\PY{n}{temp}\PY{p}{)}\PY{p}{)}\PY{p}{)}
    \PY{k}{return} \PY{n+nb}{int}\PY{p}{(}\PY{n}{temp}\PY{p}{)}
\PY{n+nb}{print}\PY{p}{(}\PY{l+s+s2}{\PYZdq{}}\PY{l+s+s2}{MDR of 86 : }\PY{l+s+s2}{\PYZdq{}}\PY{p}{,}\PY{n}{MDR}\PY{p}{(}\PY{l+m+mi}{86}\PY{p}{)}\PY{p}{)}
\PY{n+nb}{print}\PY{p}{(}\PY{l+s+s2}{\PYZdq{}}\PY{l+s+s2}{MDR of 341 : }\PY{l+s+s2}{\PYZdq{}}\PY{p}{,}\PY{n}{MDR}\PY{p}{(}\PY{l+m+mi}{341}\PY{p}{)}\PY{p}{)}
\PY{k}{def} \PY{n+nf}{MPersistence}\PY{p}{(}\PY{n}{n}\PY{p}{)}\PY{p}{:}
    \PY{n}{temp} \PY{o}{=} \PY{n+nb}{str}\PY{p}{(}\PY{n}{n}\PY{p}{)}
    \PY{n}{pers} \PY{o}{=} \PY{l+m+mi}{0}
    \PY{k}{while} \PY{n+nb}{len}\PY{p}{(}\PY{n}{temp}\PY{p}{)} \PY{o}{\PYZgt{}} \PY{l+m+mi}{1}\PY{p}{:}
        \PY{n}{temp} \PY{o}{=} \PY{n+nb}{str}\PY{p}{(}\PY{n}{prodDigits}\PY{p}{(}\PY{n+nb}{int}\PY{p}{(}\PY{n}{temp}\PY{p}{)}\PY{p}{)}\PY{p}{)}
        \PY{n}{pers} \PY{o}{+}\PY{o}{=} \PY{l+m+mi}{1}
    \PY{k}{return} \PY{n}{pers}
\PY{n+nb}{print}\PY{p}{(}\PY{l+s+s2}{\PYZdq{}}\PY{l+s+s2}{MPersistence of 86 : }\PY{l+s+s2}{\PYZdq{}}\PY{p}{,}\PY{n}{MPersistence}\PY{p}{(}\PY{l+m+mi}{86}\PY{p}{)}\PY{p}{)}
\PY{n+nb}{print}\PY{p}{(}\PY{l+s+s2}{\PYZdq{}}\PY{l+s+s2}{MPersistence of 341 : }\PY{l+s+s2}{\PYZdq{}}\PY{p}{,}\PY{n}{MPersistence}\PY{p}{(}\PY{l+m+mi}{341}\PY{p}{)}\PY{p}{)}
\end{Verbatim}
\end{tcolorbox}

    \begin{Verbatim}[commandchars=\\\{\}]
MDR of 86 :  6
MDR of 341 :  2
MPersistence of 86 :  3
MPersistence of 341 :  2
    \end{Verbatim}

    \hypertarget{write-a-function-sumpdivisors-that-finds-the-sum-of-proper-divisors-of-a-number.-proper-divisors-of-a-number-are-those-numbers-by-which-the-number-is-divisible-except-the-number-itself.-for-example-proper-divisors-of-36-are-1-2-3-4-6-9-18}{%
\section{9. Write a function sumPdivisors() that finds the sum of proper
divisors of a number. Proper divisors of a number are those numbers by
which the number is divisible, except the number itself. For example
proper divisors of 36 are 1, 2, 3, 4, 6, 9,
18}\label{write-a-function-sumpdivisors-that-finds-the-sum-of-proper-divisors-of-a-number.-proper-divisors-of-a-number-are-those-numbers-by-which-the-number-is-divisible-except-the-number-itself.-for-example-proper-divisors-of-36-are-1-2-3-4-6-9-18}}

    \begin{tcolorbox}[breakable, size=fbox, boxrule=1pt, pad at break*=1mm,colback=cellbackground, colframe=cellborder]
\prompt{In}{incolor}{17}{\boxspacing}
\begin{Verbatim}[commandchars=\\\{\}]
\PY{k}{def} \PY{n+nf}{sumPdivisors}\PY{p}{(}\PY{n}{n}\PY{p}{)} \PY{p}{:} 
    \PY{n+nb}{sum} \PY{o}{=} \PY{l+m+mi}{0}
    \PY{k}{for} \PY{n}{i} \PY{o+ow}{in} \PY{n+nb}{range}\PY{p}{(}\PY{l+m+mi}{2}\PY{p}{,} \PY{n+nb}{int}\PY{p}{(}\PY{n}{n}\PY{o}{*}\PY{o}{*}\PY{l+m+mf}{0.5}\PY{p}{)}\PY{o}{+} \PY{l+m+mi}{1}\PY{p}{)} \PY{p}{:} 
        \PY{k}{if} \PY{p}{(}\PY{n}{n} \PY{o}{\PYZpc{}} \PY{n}{i} \PY{o}{==} \PY{l+m+mi}{0}\PY{p}{)} \PY{p}{:}
            \PY{k}{if} \PY{p}{(}\PY{n}{i} \PY{o}{==} \PY{n+nb}{int}\PY{p}{(}\PY{n}{n} \PY{o}{/} \PY{n}{i}\PY{p}{)}\PY{p}{)} \PY{p}{:} 
                \PY{n+nb}{sum} \PY{o}{=} \PY{n+nb}{sum} \PY{o}{+} \PY{n}{i} 
            \PY{k}{else} \PY{p}{:} 
                \PY{n+nb}{sum} \PY{o}{=} \PY{n+nb}{sum} \PY{o}{+} \PY{p}{(}\PY{n}{i} \PY{o}{+} \PY{n+nb}{int}\PY{p}{(}\PY{n}{n} \PY{o}{/} \PY{n}{i}\PY{p}{)}\PY{p}{)} 
    \PY{k}{return} \PY{p}{(}\PY{n+nb}{sum} \PY{o}{+} \PY{l+m+mi}{1}\PY{p}{)}
\PY{n+nb}{print}\PY{p}{(}\PY{l+s+s2}{\PYZdq{}}\PY{l+s+s2}{sum of proper divisors of 36 is :}\PY{l+s+s2}{\PYZdq{}}\PY{p}{,}\PY{n}{sumPdivisors}\PY{p}{(}\PY{l+m+mi}{36}\PY{p}{)}\PY{p}{)}
\end{Verbatim}
\end{tcolorbox}

    \begin{Verbatim}[commandchars=\\\{\}]
sum of proper divisors of 36 is : 55
    \end{Verbatim}

    \hypertarget{a-number-is-called-perfect-if-the-sum-of-proper-divisors-of-that-number-is-equal-to-the-number.-for-example-28-is-perfect-number-since-12471428.-write-a-program-to-print-all-the-perfect-numbers-in-a-given-range}{%
\section{10. A number is called perfect if the sum of proper divisors of
that number is equal to the number. For example 28 is perfect number,
since 1+2+4+7+14=28. Write a program to print all the perfect numbers in
a given
range}\label{a-number-is-called-perfect-if-the-sum-of-proper-divisors-of-that-number-is-equal-to-the-number.-for-example-28-is-perfect-number-since-12471428.-write-a-program-to-print-all-the-perfect-numbers-in-a-given-range}}

    \begin{tcolorbox}[breakable, size=fbox, boxrule=1pt, pad at break*=1mm,colback=cellbackground, colframe=cellborder]
\prompt{In}{incolor}{18}{\boxspacing}
\begin{Verbatim}[commandchars=\\\{\}]
\PY{k}{def} \PY{n+nf}{perfectNumbers}\PY{p}{(}\PY{n+nb}{min}\PY{p}{,}\PY{n+nb}{max}\PY{p}{)}\PY{p}{:}
    \PY{k}{for} \PY{n}{i} \PY{o+ow}{in} \PY{n+nb}{range}\PY{p}{(}\PY{n+nb}{min}\PY{p}{,}\PY{n+nb}{max}\PY{p}{)}\PY{p}{:}
        \PY{k}{if}\PY{p}{(}\PY{n}{i} \PY{o}{==} \PY{n}{sumPdivisors}\PY{p}{(}\PY{n}{i}\PY{p}{)}\PY{p}{)}\PY{p}{:}
            \PY{n+nb}{print}\PY{p}{(}\PY{n}{i}\PY{p}{)}
\PY{n+nb}{print}\PY{p}{(}\PY{l+s+s2}{\PYZdq{}}\PY{l+s+s2}{Perfect numbers in range 1\PYZhy{}1000 are :}\PY{l+s+s2}{\PYZdq{}}\PY{p}{)}
\PY{n}{perfectNumbers}\PY{p}{(}\PY{l+m+mi}{1}\PY{p}{,}\PY{l+m+mi}{1000}\PY{p}{)}
\end{Verbatim}
\end{tcolorbox}

    \begin{Verbatim}[commandchars=\\\{\}]
Perfect numbers in range 1-1000 are :
1
6
28
496
    \end{Verbatim}

    \hypertarget{two-different-numbers-are-called-amicable-numbers-if-the-sum-of-the-proper-divisors-of-each-is-equal-to-the-other-number.-for-example-220-and-284-are-amicable-numbers.sum-of-proper-divisors-of-220-1-2-4-5-10-11-20-22-44-55-110-284-sum-of-proper-divisors-of-284-1-2-4-71-142-220-.-write-a-function-to-print-pairs-of-amicable-numbers-in-a-range}{%
\section{11. Two different numbers are called amicable numbers if the
sum of the proper divisors of each is equal to the other number. For
example 220 and 284 are amicable numbers.Sum of proper divisors of 220 =
1 + 2 + 4 + 5 + 10 + 11 + 20 + 22 + 44 +55 + 110 = 284 Sum of proper
divisors of 284 = 1 + 2 + 4 + 71 + 142 = 220 . Write a function to print
pairs of amicable numbers in a
range}\label{two-different-numbers-are-called-amicable-numbers-if-the-sum-of-the-proper-divisors-of-each-is-equal-to-the-other-number.-for-example-220-and-284-are-amicable-numbers.sum-of-proper-divisors-of-220-1-2-4-5-10-11-20-22-44-55-110-284-sum-of-proper-divisors-of-284-1-2-4-71-142-220-.-write-a-function-to-print-pairs-of-amicable-numbers-in-a-range}}

    \begin{tcolorbox}[breakable, size=fbox, boxrule=1pt, pad at break*=1mm,colback=cellbackground, colframe=cellborder]
\prompt{In}{incolor}{19}{\boxspacing}
\begin{Verbatim}[commandchars=\\\{\}]
\PY{k}{def} \PY{n+nf}{amicablePairs}\PY{p}{(}\PY{n}{x}\PY{p}{,} \PY{n}{y}\PY{p}{)} \PY{p}{:} 
  
    \PY{k}{if} \PY{p}{(}\PY{n}{sumPdivisors}\PY{p}{(}\PY{n}{x}\PY{p}{)} \PY{o}{!=} \PY{n}{y}\PY{p}{)} \PY{p}{:} 
        \PY{k}{return} \PY{k+kc}{False}
          
    \PY{k}{return} \PY{p}{(}\PY{n}{sumPdivisors}\PY{p}{(}\PY{n}{y}\PY{p}{)} \PY{o}{==} \PY{n}{x}\PY{p}{)}

\PY{k}{def} \PY{n+nf}{amicableNumbers}\PY{p}{(}\PY{n+nb}{min}\PY{p}{,}\PY{n+nb}{max}\PY{p}{)}\PY{p}{:}
    \PY{k}{for} \PY{n}{i} \PY{o+ow}{in} \PY{n+nb}{range}\PY{p}{(}\PY{n+nb}{min}\PY{p}{,}\PY{n+nb}{max}\PY{p}{)}\PY{p}{:}
        \PY{k}{for} \PY{n}{j} \PY{o+ow}{in} \PY{n+nb}{range}\PY{p}{(}\PY{n}{i}\PY{o}{+}\PY{l+m+mi}{1}\PY{p}{,}\PY{n+nb}{max}\PY{p}{)}\PY{p}{:}
            \PY{k}{if}\PY{p}{(}\PY{n}{amicablePairs}\PY{p}{(}\PY{n}{i}\PY{p}{,}\PY{n}{j}\PY{p}{)}\PY{p}{)}\PY{p}{:}
                \PY{n+nb}{print}\PY{p}{(}\PY{n}{i}\PY{p}{,}\PY{n}{j}\PY{p}{)}

\PY{n+nb}{print}\PY{p}{(}\PY{l+s+s2}{\PYZdq{}}\PY{l+s+s2}{Amicable number in range 1\PYZhy{}2000 :}\PY{l+s+s2}{\PYZdq{}} \PY{p}{)}
\PY{n}{amicableNumbers}\PY{p}{(}\PY{l+m+mi}{1}\PY{p}{,}\PY{l+m+mi}{2000}\PY{p}{)}
\end{Verbatim}
\end{tcolorbox}

    \begin{Verbatim}[commandchars=\\\{\}]
Amicable number in range 1-2000 :
220 284
1184 1210
    \end{Verbatim}

    \hypertarget{write-a-program-which-can-filter-odd-numbers-in-a-list-by-using-filter-function}{%
\section{12. Write a program which can filter odd numbers in a list by
using filter
function}\label{write-a-program-which-can-filter-odd-numbers-in-a-list-by-using-filter-function}}

    \begin{tcolorbox}[breakable, size=fbox, boxrule=1pt, pad at break*=1mm,colback=cellbackground, colframe=cellborder]
\prompt{In}{incolor}{20}{\boxspacing}
\begin{Verbatim}[commandchars=\\\{\}]
\PY{n}{list1} \PY{o}{=} \PY{p}{[}\PY{l+m+mi}{101}\PY{p}{,}\PY{l+m+mi}{10}\PY{p}{,}\PY{l+m+mi}{1}\PY{p}{,}\PY{l+m+mi}{24}\PY{p}{,}\PY{l+m+mi}{2}\PY{p}{,}\PY{l+m+mi}{3}\PY{p}{,}\PY{l+m+mi}{4}\PY{p}{,}\PY{l+m+mi}{5}\PY{p}{,}\PY{l+m+mi}{9}\PY{p}{,}\PY{l+m+mi}{13}\PY{p}{,}\PY{l+m+mi}{22}\PY{p}{,}\PY{l+m+mi}{17}\PY{p}{,}\PY{l+m+mi}{99}\PY{p}{]}
\PY{k}{def} \PY{n+nf}{Odd}\PY{p}{(}\PY{n}{x}\PY{p}{)}\PY{p}{:}
    \PY{k}{if} \PY{n}{x} \PY{o}{\PYZpc{}} \PY{l+m+mi}{2} \PY{o}{!=} \PY{l+m+mi}{0}\PY{p}{:}
        \PY{k}{return} \PY{k+kc}{True}
    \PY{k}{else}\PY{p}{:}
        \PY{k}{return} \PY{k+kc}{False}

\PY{n}{odd\PYZus{}list} \PY{o}{=} \PY{n+nb}{filter}\PY{p}{(}\PY{n}{Odd} \PY{p}{,} \PY{n}{list1}\PY{p}{)}
\PY{k}{for} \PY{n}{x} \PY{o+ow}{in} \PY{n}{odd\PYZus{}list}\PY{p}{:}
    \PY{n+nb}{print}\PY{p}{(}\PY{n}{x}\PY{p}{)}
\end{Verbatim}
\end{tcolorbox}

    \begin{Verbatim}[commandchars=\\\{\}]
101
1
3
5
9
13
17
99
    \end{Verbatim}

    \hypertarget{write-a-program-which-can-map-to-make-a-list-whose-elements-are-cube-of-elements-in-a-given-list}{%
\section{13. Write a program which can map() to make a list whose
elements are cube of elements in a given
list}\label{write-a-program-which-can-map-to-make-a-list-whose-elements-are-cube-of-elements-in-a-given-list}}

    \begin{tcolorbox}[breakable, size=fbox, boxrule=1pt, pad at break*=1mm,colback=cellbackground, colframe=cellborder]
\prompt{In}{incolor}{21}{\boxspacing}
\begin{Verbatim}[commandchars=\\\{\}]
\PY{n}{list2} \PY{o}{=} \PY{p}{[}\PY{l+m+mi}{1}\PY{p}{,}\PY{l+m+mi}{2}\PY{p}{,}\PY{l+m+mi}{3}\PY{p}{,}\PY{l+m+mi}{4}\PY{p}{,}\PY{l+m+mi}{5}\PY{p}{,}\PY{l+m+mi}{6}\PY{p}{]}
\PY{k}{def} \PY{n+nf}{cube}\PY{p}{(}\PY{n}{x}\PY{p}{)}\PY{p}{:}
    \PY{k}{return} \PY{n}{x}\PY{o}{*}\PY{n}{x}\PY{o}{*}\PY{n}{x}
\PY{n}{cube\PYZus{}list} \PY{o}{=} \PY{n+nb}{map}\PY{p}{(}\PY{n}{cube}\PY{p}{,} \PY{n}{list2}\PY{p}{)}
\PY{k}{for} \PY{n}{x} \PY{o+ow}{in} \PY{n}{cube\PYZus{}list}\PY{p}{:}
    \PY{n+nb}{print}\PY{p}{(}\PY{n}{x}\PY{p}{)}
\end{Verbatim}
\end{tcolorbox}

    \begin{Verbatim}[commandchars=\\\{\}]
1
8
27
64
125
216
    \end{Verbatim}

    \hypertarget{write-a-program-which-can-map-and-filter-to-make-a-list-whose-elements-are-cube-of-even-number-in-a-given-list}{%
\section{14. Write a program which can map() and filter() to make a list
whose elements are cube of even number in a given
list}\label{write-a-program-which-can-map-and-filter-to-make-a-list-whose-elements-are-cube-of-even-number-in-a-given-list}}

    \begin{tcolorbox}[breakable, size=fbox, boxrule=1pt, pad at break*=1mm,colback=cellbackground, colframe=cellborder]
\prompt{In}{incolor}{22}{\boxspacing}
\begin{Verbatim}[commandchars=\\\{\}]
\PY{n}{list3} \PY{o}{=} \PY{p}{[}\PY{l+m+mi}{101}\PY{p}{,}\PY{l+m+mi}{10}\PY{p}{,}\PY{l+m+mi}{1}\PY{p}{,}\PY{l+m+mi}{24}\PY{p}{,}\PY{l+m+mi}{2}\PY{p}{,}\PY{l+m+mi}{3}\PY{p}{,}\PY{l+m+mi}{4}\PY{p}{,}\PY{l+m+mi}{5}\PY{p}{,}\PY{l+m+mi}{9}\PY{p}{,}\PY{l+m+mi}{13}\PY{p}{,}\PY{l+m+mi}{22}\PY{p}{,}\PY{l+m+mi}{17}\PY{p}{,}\PY{l+m+mi}{99}\PY{p}{]}

\PY{k}{def} \PY{n+nf}{Even}\PY{p}{(}\PY{n}{x}\PY{p}{)}\PY{p}{:}
    \PY{k}{if} \PY{n}{x} \PY{o}{\PYZpc{}} \PY{l+m+mi}{2} \PY{o}{==} \PY{l+m+mi}{0}\PY{p}{:}
        \PY{k}{return} \PY{k+kc}{True}
    \PY{k}{else}\PY{p}{:}
        \PY{k}{return} \PY{k+kc}{False}

\PY{n}{even\PYZus{}list} \PY{o}{=} \PY{n+nb}{filter}\PY{p}{(}\PY{n}{Even}\PY{p}{,} \PY{n}{list3}\PY{p}{)}
\PY{n}{even\PYZus{}cube\PYZus{}list} \PY{o}{=} \PY{n+nb}{map}\PY{p}{(}\PY{n}{cube}\PY{p}{,} \PY{n}{even\PYZus{}list}\PY{p}{)} \PY{c+c1}{\PYZsh{}using cube function made in problem 13}

\PY{k}{for} \PY{n}{x} \PY{o+ow}{in} \PY{n}{even\PYZus{}cube\PYZus{}list}\PY{p}{:}
    \PY{n+nb}{print}\PY{p}{(}\PY{n}{x}\PY{p}{)}
\end{Verbatim}
\end{tcolorbox}

    \begin{Verbatim}[commandchars=\\\{\}]
1000
13824
8
64
10648
    \end{Verbatim}


    % Add a bibliography block to the postdoc
    
    
    
\end{document}
